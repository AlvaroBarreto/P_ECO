\chapter[S\'intesis de almid\'on por el proceso de la fotos\'intesis]{Almid\'on}

\begin{Huge}
	\begin{center}
		\textbf{S\'intesis de almid\'on por el proceso de la fotos\'intesis.}
	\end{center}
\end{Huge}

\section{Introducci\'on}

En la mayor\'ia de las especies, la sacarosa es la principal forma de carbohidrato que se transporta a trav\'es de la planta por el floema. El almid\'on es una reserva de carbohidratos estables e insolubles que est\'a presente en casi todas las plantas. Tanto el almid\'on como la sacarosa se sintetizan a partir de la triosa fosfato que se genera en el ciclo de Calvin. 

El cloroplasto es el sitio de sintesis de almid\'on en las hojas. El almid\'on se sintetiza a partir de la triosa fosfato de la fructosa-1, 6-bisfosfato. El intermediario glucosa-1-fosfato se convierte en ADP-glucosa a través de la ADP-glucosa pirofosforilasa en una reacción que requiere ATP y genera pirofosfato. Como en muchas reacciones biosintéticas, el pirofosfato
se hidroliza a través de una pirofosfatasa inorgánica específica a dos mol\'eculas de ortofosfato, impulsando así la reacción hacia la síntesis de ADP-glucosa. Finalmente, la fracción de glucosa de la ADP-glucosa se transfiere al extremo no reductor de la glucosa terminal de una cadena de almidón en crecimiento, completando as\'i la secuencia de reacci\'on.

\subsection{Principio}

Las concentraciones relativas de ortofosfato y triosa fosfato son los principales factores que controlan si el carbono fijado fotosintéticamente se reparte como almidón en el cloroplasto o como sacarosa en el citosol. El ortofosfato y la triosa fosfato controlan la actividad de varias enzimas reguladoras en las vías biosintéticas de la sacarosa y el almidón. La enzima del cloroplasto ADP-glucosa pirofosforilasa es la enzima clave que regula la síntesis de almidón a partir de glucosa-1-fosfato. Esta enzima es estimulada por el 3-fosfoglicerato e inhibida por el ortofosfato. En los cloroplastos iluminados que sintetizan activamente el almidón se suele encontrar una alta proporción de concentración de 3-fosfoglicerato respecto al ortofosfato. Condiciones recíprocas prevalecen en la oscurida.

\section{Objetivo general}

Evaluar la presencia de almid\'on en hojas en condici\'on de oscuridad e iluminaci\'on natural. 

\section{Objetivos espec\'ificos}

\begin{enumerate}
	\item Determinar la presencia de granos de almid\'on en hojas en condici\'on de ausencia de luz solar.
	\item Determinar la presencia de granos de almid\'on en plantas con iluminaci\'on natural.
\end{enumerate}

\section{Materiales}

\subsection{Material requerido}

\begin{enumerate}
	\item Dos vasos de precipitado de 500 mL 
	\item Dos cajas petri
	\item Ba\~no Mar\'ia
\end{enumerate}

\subsection{Material por grupo}

\begin{enumerate}
	\item Un litro de alcohol et\'ilico al 96\%
	\item Soluci\'on de lugol al 10\%
	\item Cuatro microscopios estereoscópicos
\end{enumerate}

\section{Metodolog\'ia}

\subsection{Preparaci\'on de material}

Conseguir dos plantas en un vivero. Una planta se mantendr\'a en oscuridad por dos d\'ias, cubriéndola con una caja de cart\'on. La otra planta se mantendr\'a en un lugar iluminado. 

\subsection{Procedimiento}

En los vasos de precipitado se agregar\'an aproximadamente 100 mL de alcohol et\'ilico. Luego ser\'an colocados en un ba\~no Mar\'ia con agua a punto de ebullici\'on. En uno de los vasos de precipitado se van a sumergir tres hojas de la planta que se mantuvo en la oscuridad y en el otro las hojas de la planta que se mantuvo en un lugar iluminado. Posteriormente, se mantendr\'an en ba\~no Mar\'ia  por 20 min o hasta que las hojas se decoloren. 

Se retiran las hojas y se lavan con agua caliente. Luego, se ponen las hojas en cajas Petri y se agrega soluci\'on de lugol hasta cubrir las hojas. 

Observar las hojas en el microscopio estereosc\'opico y contar las manchas blancas que se formaron.

%Preparar un trozo fino de tubérculo de patata con la ayuda de una cuchilla o navaja. Móntelo sobre un tobogán en una gota de agua. Añadir una gota de solución de yodo (IKI) y cubrir con una tapa de vidrio. Examinar bajo el microscopio de luz. Los granos de almidón se tiñen de azul. Los granos de almidón viejos se tiñen de negro azulado, mientras que el almidón fresco da un color rojo o púrpura.

\section{Resultados}

Tomar fotograf\'ias a las hojas y presentar el conteo de n\'umero manchas por hoja que obtuvieron las plantas que se mantuvieron en un lugar iluminado versus ausencia de luz.

\section{Cuestionario}

\begin{enumerate}
	\item Explique la relaci\'on entre el n\'umero de manchas por hoja en condiciones de oscuridad.
	%\item \textquestiondown Cu\'al es la funci\'on del almid\'on?

\end{enumerate}

