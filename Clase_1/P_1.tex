\documentclass[12pt, aspectratio=169]{beamer}
\usepackage[utf8]{inputenc}
\usepackage[T1]{fontenc}
\usepackage{lmodern}
\usepackage[spanish, es-nodecimaldot]{babel}
\uselanguage{spanish}
\languagepath{spanish}
\usepackage{amsmath}
\usepackage{amsfonts}
\usepackage{amssymb}
\usepackage{siunitx}
\usepackage{graphicx}
\usepackage{subcaption}
\usepackage{ragged2e}
\renewcommand{\arraystretch}{1.2}
\author{Alvaro Barreto}
\useinnertheme{rounded} %rectangles, rounded, inmargin, circles
\usetheme{Dresden} %Hannover AnnArbor CambridgeUS Dresden
%\usecolortheme{seahorse}

\makeatletter
\setbeamertemplate{footline}
{
	\leavevmode%
	\hbox{%
		\begin{beamercolorbox}[wd=.333333\paperwidth,ht=2.25ex,dp=1ex,center]{author in head/foot}%
			\usebeamerfont{author in head/foot}\insertshortauthor
		\end{beamercolorbox}%
		\begin{beamercolorbox}[wd=.333333\paperwidth,ht=2.25ex,dp=1ex,center]{title in head/foot}%
			\usebeamerfont{title in head/foot}\insertshorttitle
		\end{beamercolorbox}%
		\begin{beamercolorbox}[wd=.333333\paperwidth,ht=2.25ex,dp=1ex,right]{date in head/foot}%
			\usebeamerfont{date in head/foot}\insertshortdate{}\hspace*{2em}
			\insertframenumber{} / \inserttotalframenumber\hspace*{2ex} 
	\end{beamercolorbox}}%
	\vskip0pt%
}
\makeatother

\setbeamertemplate{blocks}[rounded][shadow=false]
\definecolor{LightViolet}{RGB}{230,220,255}
\setbeamercolor{frametitle}{bg=LightViolet}

\definecolor{NormalBlue}{RGB}{200,200,255}
\setbeamercolor{block title}{bg=NormalBlue}

\definecolor{LightBlue}{RGB}{220,220,255}
\setbeamercolor{block body}{bg=LightBlue}

\setbeamercolor{block body example}{bg=red!20!white}
\setbeamercolor{block title example}{fg=red, bg=red!40!white}

\newcommand{\col}{\column{0.5\textwidth}}
\newcommand{\just}{\justifying}

\begin{document}

% Presentaci\'on
%\author{Alvaro Barreto}
\title{Matem\'aticas para el laboratorio}
\subtitle{Soluciones, mezclas y medios}
\titlegraphic{
	\includegraphics[width=3.5cm]{ENES}
}

\begin{frame}[plain]
	\titlepage
\end{frame}

\section{Notaci\'on cient\'ifica y prefijos m\'etricos}

\begin{frame}{Tabla de contenidos}
	\tableofcontents
\end{frame}

\begin{frame}[c]{D\'igitos significativos} \framesubtitle{Redondeo de d\'igitos}

\begin{columns}
	\col
	\just
	Cuando se trabaja con mediciones, debemos tener en cuenta la 
	\alert{precisi\'on}. Por lo tanto, para cualquier c\'alculo se debe redondear para reflejar el nivel m\'as bajo de precisi\'on. 
		
	\col
	
	\begin{block}{Gu\'ia} \just
		Para toda operaci\'on aritm\'etica, el resultado debe ser redondeado para que tenga el mismo n\'umero de d\'igitos significativos decimales que el valor con el menor n\'umero de decimales utilizados en el c\'alculo.
	\end{block}
	
\end{columns}

\end{frame}

\begin{frame}[t]{Ejemplo} %[allowframebreaks, allowdisplaybreaks, t]
	Exprese los resultados de los siguientes c\'alculos tomando en cuenta la gu\'ia anterior.
	
	\vspace{1.5ex}
		
	\begin{columns}
		\col
		
		\textbf{A)}
		$$0.2385~\text{g} + 25.8~ \text{g} = 26.0385~\text{g}$$
		
		Redondeando:
		
		$$26.0385~\text{g} \approx 26~\text{g} $$
		
		\col
		\textbf{B)}
			$$5.5~\text{cm} \times 3.356 ~\text{cm} = 18.458~\text{cm}^2$$
		
		Redondeando:
		
		$$18.458~\text{cm}^2 \approx 18.5~\text{cm}^2$$
						
	\end{columns}
		
\end{frame}

\begin{frame}{Notaci\'on cient\'ifica}
	Por lo general, los exponentes con base 10 se utilizan en la notaci\'on cient\'ifica para expresar n\'umeros en forma abreviada.
	
	$$10^{-3} = \frac{1}{10^3} = \frac{1}{1,000} = 0.001$$
	
\end{frame}

\begin{frame}
	
	Se tiene que colocar un signo de multiplicaci\'on y el n\'umero 10 a la derecha de los d\'igitos enteros significativos. Luego, un exponente indicar\'a el n\'umero de posiciones que el punto se mover\'a, es decir, n\'umeros superiores a 10, se utiliza un exponente positivo, y para n\'umeros menores a 1 el exponente es negativo	
\end{frame}

\begin{frame}[t]{Ejemplo}
	\begin{columns}
		\col
		\textbf{A)}
		$$4,003,000,000.0 = 4.003 \times 10^9 $$
		\textbf{B)}
		$$65.0 = 6.5 \times 10^1$$
		\textbf{C)}
		$$60.29 \times 10^{22} = 6.029 \times 10^{23}$$
				
		\col
		\textbf{D)}
		$$0.0000000025 = 2.5 \times 10^{-8}$$
		\textbf{F)}
		$$0.0002001 = 2.001 \times 10^{-4}$$
		\textbf{G)}
		$$528.69 \times 10^{-7} = 5.2869 \times 10^{-5}$$
				
	\end{columns}
	
\end{frame}

\begin{frame}{Prefijos m\'etricos} 
	
	\begin{table}
		\caption{Prefijos m\'etricos}
		\begin{tabular}{lcc}
			\hline
			Prefijo & Abreviatura & Exponente \\
			\hline
			giga- & G & $10^9$ \\
			mega- & M & $10^6$ \\
			kilo- & K & $10^3$ \\
			mili- & m  & $10^{-3}$ \\
			micro- & $\mu$  & $10^{-6}$ \\
			nano- & n & $10^{-9}$ \\
			\hline
		\end{tabular}
	\end{table}
	
\end{frame}

\begin{frame}{Factor de conversi\'on}
	El factor de conversi\'on es una relaci\'on num\'erica igual a 1.

	\vspace{1.2ex}	

	\begin{columns}
		\col
		$$\frac{1 \times 10^6~ \mu L}{1~ \text{L}}$$
			
		\col
		$$\frac{1~ \text{L}}{1 \times 10^6~ \mu L}$$
		
	\end{columns}
	
	\vspace{1.2ex}
	
	Cuando se realizan conversiones aparecen t\'erminos similares en el numerador o denominador, pueden ser cancelados. Por ejemplo, convertir $3 \times 10^{-4}~ \text{L}$ a microlitros.
	
	
\end{frame}

\begin{frame}[t]{Ejemplo I}
	\begin{columns}
		\col
		$$\text{\textbf{X}}~ \mu \text{L} = 3 \times 10^{-4}~ \text{L}$$
		
		$$\text{\textbf{X}}~ \mu \text{L} = 3 \times 10^{-4}~ \text{L} \times \frac{1 \times 10^6~ \mu L}{1~ \text{L}} $$
		
		
		\col
		Vamos a utilizar el factor de conversi\'on en relaci\'on con litros y microlitros. Los t\'erminos id\'enticos en el numerador y denominador se cancelan. 
				
	\end{columns}	
	
	\vspace{1.2ex}
	
	\begin{columns}
		\col
		$$\text{\textbf{X}}~ \mu \text{L} = (3 \times 1)(10^{-4} \times 10^{6}) \mu \text{L}$$
		\vspace{-1.4ex}
		$$\text{\textbf{X}}~ \mu \text{L} = 3 \times 10^{-4+6} \mu \text{L} $$
		\vspace{-1.4ex}
		$$3 \times 10^2\mu \text{L}$$
		\col	
		Se agrupan los t\'erminos similares, y se multiplican los numeradores. Por lo tanto:
		
		$$3 \times 10^2\mu \text{L} = 3 \times 10^{-4} ~\text{L}$$
		
	\end{columns}
	
\end{frame}


\begin{frame}[t]{Aritm\'etica de n\'umeros en notaci\'on cient\'ifica}
		
	En la suma y resta se transforman los n\'umeros para que tengan el mismo exponente en base 10.
	
	\vspace{1.5ex}
	
	\begin{columns}
		\col
		\onslide<2->{
		\textbf{Suma:}
		$$(5 \times 10^3) + (2 \times 10^3)$$
		$$(5 + 2) \times 10^3$$
		$$7 \times 10^3$$}
		\col
		\onslide<3->{
		\textbf{Resta: }
		$$(8 \times 10^{-2}) - (\num{6e-3})$$
		$$(\num{8e-2}) - (\num{0.6e-2})$$
		$$(8-0.6) \times 10^{-2}$$
		$$\num{7.4e-2}$$
		}		
	\end{columns}
	
\end{frame}

\begin{frame}[t]
	En la multiplicaci\'on y divisi\'on se aplica la regla del producto y cociente. Adem\'as de la propiedad conmutativa y asociativa.
	%propiedad conmutativa ($3 \times 2 = 2 \times 3$)
	%asociativa [$2 \times (3 \times 4) = (2 \times 3) \times 4 $]
	\vspace{1.5ex}
	\begin{columns}
		\col
		\onslide<2->{
		\textbf{Multiplicaci\'on:}
		$$(\num{4e4}) \times (\num{3e2})$$
		$$(4 \times 3) \times (10^4 \times 10^2)$$
		$$\num{12e6}$$
		$$\num{1.2e7}$$
		}
		\col
		\onslide<3->{
		\textbf{Divisi\'on:}
		$$\frac{\num{8.2e4}}{\num{3.6e-6}}$$
		$$\frac{8.2}{3.6} \times 10^{4-(-6)}$$
		$$\num{2.3e12}$$
		}
	\end{columns}
	
\end{frame}

\section{Soluciones, mezclas y medios}

\begin{frame}{Tabla de contenidos}
	\tableofcontents[currentsection]
\end{frame}

\begin{frame}{Diluciones}

	\begin{block}{Definici\'on}
		\textbf{Concentraci\'on} es la cantidad de una determinada sustancia en un volumen dado.
	\end{block}
	
	$$Concentraci\acute{o}n = \frac{Cantidad}{Volumen}$$
	\just
	De manera general en los laboratorios se preparan soluciones concentradas (soluci\'on madre) de los reactivos de uso com\'un. En algunas ocasiones se utilizan las soluciones madre para elaborar concentraciones menos concentradas. Por lo tanto, se realiza una \textbf{diluci\'on}.
	
\end{frame}

\begin{frame}[allowframebreaks, allowdisplaybreaks] {C\'alculo de la concentraci\'on de un reactivo diluido} %M\'ultiples diapositivas
	\framesubtitle{M\'etodo I}
	Se puede usar la ecuaci\'on $C_1V_1 = C_2V_2$
	
	Donde:
	\begin{itemize}
		\item $C_1 =$ concentraci\'on inicial de la soluci\'on madre ("stock").
		\item $V_1 =$ cantidad de soluci\'on madre para realizar la diluci\'on, es decir, \textbf{la cantidad que vamos a ocupar}.
		\item $C_2 =$ la concentraci\'on de la muestra diluida.
		\item $V_2 =$ el volumen total (\textbf{final}) de la muestra diluida.  
	\end{itemize}
	
	Por ejemplo, se tiene una soluci\'on madre de sacarosa (az\'ucar) al 20\%, y se requiere 5 mL de una soluci\'on al 3\% de sacarosa. \textquestiondown Cu\'antos $\mu$L de soluci\'on madre se necesitan?
	
	\textbf{Paso 1:}
	
	Convertir los 5 mL a $\mu$L
	
	$$\frac{1~ \text{L}}{\num{1e3} ~\text{mL}} \times \frac{\num{1e6}~\mu \text{L}}{1~ \text{L}} \times 5 ~\text{mL} = (1 \times 5) \times 10^{6-3} ~\mu \text{L} = \num{5e3} ~\mu \text{L} $$ 
	
	\textbf{Paso 2:}
	Aplicando la ecuaci\'on
	$$C_1V_1 = C_2V_2$$
	$$20\% \times V_1 = 3\% \times (\num{5e3})~\mu \text{L}$$	
	$$V_1 = \frac{3\% \times (\num{5e3}~\mu \text{L})}{\num{2e1}\%} = \frac{15}{2} \times 10^{3-1} ~\mu \text{L} = \num{7.5e2} ~\mu \text{L}$$
	
	\textbf{Paso 3:} 
	Aforando \\
	Se necesitan \num{7.5e2} $\mu \text{L}$ de sacarosa al 20\% m\'as \num{4.25e3}$\mu \text{L}$ de agua ($\num{5e3}-\num{7.5e2}$). 
		
\end{frame}

\begin{frame}[allowframebreaks, allowdisplaybreaks] {C\'alculo de la concentraci\'on de un reactivo diluido}
	\framesubtitle{M\'etodo II}
	El m\'etodo de \textbf{an\'alisis dimensional} incorpora el factor de conversi\'on como parte de la f\'ormula y la ecuaci\'on es:
	$$conc. ~ de~ partida \times factor ~de~ conversi\acute{o}n \times \frac{volumen ~desconocido}{volumen ~final} = conc. ~deseada$$
	
	Tomando el ejemplo anterior:
	
	$$\num{2e1}\% \times \frac{1 ~\text{mL}}{\num{1e3}~\mu \text{L}} \times \frac{x ~\mu \text{L}}{5 ~\text{mL}} = 3\%$$
	
	\vspace{2ex}
	
	$$\frac{\num{2e1}\% \times x~\mu \text{L}}{\num{5e3}} = 3\%$$
	
	$$x = \frac{3\% \times \num{5e3}}{\num{2e1}\%} = \num{7.5e2}~\mu \text{L}$$
	
\end{frame}

\begin{frame}{Preparar soluciones en porcentaje}
	
	Muchos reactivos se preparan en porcentaje de un soluto disuelto en una soluci\'on. Por lo tanto, dependiendo del estado f\'isico del soluto se expresa en porcentaje peso en volumen (\%p/v) o en porcentaje volumen en volumen (\%v/v).
	
	En el porcentaje peso en volumen (\%p/v), el peso en soluto es expresado en gramos en un total de 100 mL de soluci\'on.

\end{frame}

\begin{frame}{Preparar soluciones en porcentaje}
	\framesubtitle{Ejemplo I}
	Preparar 50 mL de una soluci\'on de $NaCl$ al 8\%.
	
	$$\frac{8}{100} \times 50 = 4$$
	
	Por lo tanto, se necesitan 4 g de $NaCl$ en 50 mL de agua destilada para preparar la soluci\'on de $NaCl$ al 8\%.
	
\end{frame}

\begin{frame}{Preparar soluciones en porcentaje}
	\framesubtitle{Ejemplo II}
	Preparar 500 mL de soluci\'on de etanol al 85\%.
	
	$$\frac{85}{100} \times 500 = 425$$
	
	Por lo tanto, para prepara una soluci\'on de etanol al 85\%, se necesita 425 mL de etanol al 100\% y se agregan 75 mL de agua destilada para obtener el volumen final de 500 mL. 
	
\end{frame}

\begin{frame}{Preparar soluciones en porcentaje}
		
	Se tiene 2 mL de etanol al 95\% y se agregan 5 mL de agua destilada. \textquestiondown Cu\'al es la concentraci\'on final de la soluci\'on de etanol?
	
	$$C_1V_1 = C_2V_2$$
	$$\frac{95}{100} \times 2 \text{mL} = \frac{C_2}{100} \times 7 \text{mL}$$
	$$C_2 = \frac{95}{100} \times \frac{2 \text{mL}}{ 7 \text{mL}} \times 100 = 27$$
	
	Por lo tanto, la soluci\'on final es igual al 27\%.
	
\end{frame}

\begin{frame}{Moles y peso molecular}

	Un mol equivale a \num{6.023e23} mol\'eculas, es tambi\'en conocido como n\'umero de Avogadro. Peso molecular es el t\'ermino popular para referirse a la masa molecular. El peso molecular es equivalente a la suma de todos los pesos at\'omicos. Por ejemplo, el peso at\'omico de $NaCl$ es 58.44g por causa de la suma del peso at\'omico de $Na$ 22.99 g y del $Cl$ 35.45 g. 
	
	En la mayor\'ia de los reactivos este se encuentra en la etiqueta como \textbf{FW} (Formula Weight)

\end{frame}

\begin{frame}{Molaridad}
	
	Una soluci\'on de 1 M contiene el peso molecular (gramos) de una sustancia por 1 L de soluci\'on.
		
\end{frame}

\begin{frame}[allowframebreaks, allowdisplaybreaks]{Molaridad}
	\framesubtitle{Ejemplo}
	
	Preparar 100 mL de $NaCl$ al 0.5 M
	$$\frac{58.44 ~\text{g}}{ 1 ~\text{M}} = \frac{x ~\text{g}}{0.5 ~\text{M}}$$
	
	$$x ~\text{g} = \frac{58.44 ~\text{g}}{ 1 ~\text{M}} \times 0.5~\text{M} = 29.2 ~\text{g}$$
	
	$$\frac{29.2~\text{g}}{1000 ~\text{mL}} = \frac{x ~\text{g}}{100 ~\text{mL}}$$
	
	$$x ~\text{g} = \frac{29.2~\text{g}}{1000 ~\text{mL}} \times 100 ~\text{mL} = 2.9$$
			
\end{frame}

\end{document}